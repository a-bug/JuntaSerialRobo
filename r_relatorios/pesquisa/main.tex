\documentclass[12pt,a4paper,oneside,english, brazil]{abntex2}

%%%%%%%%%%%%%%%%%%%%%%%%%%%%%
%%PACOTES A SEREM INCLUÍDOS%%
%%%%%%%%%%%%%%%%%%%%%%%%%%%%%
\usepackage[T1]{fontenc}
\usepackage[utf8]{inputenc}
\usepackage[absolute,overlay]{textpos}
\usepackage{hyperref}
\usepackage{ragged2e}
\usepackage{hyphenat}
\usepackage{calrsfs}
\usepackage{indentfirst}
\usepackage{microtype} 
\usepackage{float}
\usepackage{caption}
\usepackage{subcaption}
\usepackage{helvet}
\usepackage[bottom]{footmisc}
\usepackage{catchfilebetweentags}
\usepackage{enumerate}
\usepackage{lscape}
\usepackage{enumitem}
\usepackage{caption}
\usepackage{titling}
\usepackage{footmisc}
\usepackage{scrextend}
\usepackage{titlesec}
\usepackage{pdfpages}
\usepackage{nomencl}

%Color packages
\usepackage{transparent}
\usepackage{xcolor}
\usepackage{tcolorbox}

%Table packages
\usepackage{tabu}
\usepackage{longtable}
\usepackage{booktabs}
\usepackage{multicol}
\usepackage{multirow}

%Equation packages
\usepackage{amsmath}
\usepackage{amsfonts}
\usepackage{amssymb}
\usepackage{amstext}
\usepackage{sansmath}
\usepackage{bm}
\usepackage{breqn}
\usepackage{mathtools}

%Image package
\usepackage{tikz}
\usepackage{graphicx}
\usepackage{geometry}


%%%%%%%%%%%%%%%%%%%%%%%%%%%%%%%%%%%%%%%%%%
%%NOME DO AUTOR, TÍTULO DO TRABALHO E ANO%
%%%%%%%%%%%%%%%%%%%%%%%%%%%%%%%%%%%%%%%%%%

\newcommand{\aaluno}{Michael Luis Weiss}
\newcommand{\atitulo} {Desenvolvimento de um sistema de comunicação serial para controle com ROS2}
\newcommand{\aano}{2023}
\newcommand{\acurso}{Ciência e Tecnologia}
\newcommand{\aorientador}{Orientador: Prof. Dr. Roberto Simoni}
% Roberto Simoni Ramal: 2632	 Sala: U264
% E-mail: roberto.emc@gmail.com
% http://buscatextual.cnpq.br/buscatextual/visualizacv.do?id=K4248505H7
\newcommand{\acoorientador}{}

%%%%%%%%%%%%%%%%%%%%%%%%%%%
%%DEFINIÇÕES DE PRÊAMBULO%%
%%%%%%%%%%%%%%%%%%%%%%%%%%%
\ExecuteMetaData[settings.tex]{preambleTag}
\pagestyle{plain}
\pagenumbering{arabic}
\begin{document}
%%%%%%%%%%%%%%%%%%%%%%%%%
%%CAPA E FOLHA DE ROSTO%%
%%%%%%%%%%%%%%%%%%%%%%%%%
\ExecuteMetaData[settings.tex]{capaTag}



\textbf{Resumo}

Alguns processos industriais dependem do controle de pose de robôs manipuladores com certa precisão, o que geralmente requer equipamentos de alta qualidade e custo. Pesquisas realizadas no laboratório LGI da UFSC utilizando processamento de imagens, uma câmera de alto desempenho e marcadores fiduciais já foram desenvolvidas e apresentaram resultados de precisão competitivos comparado à equipamentos de medição sofisticados. Neste trabalho, foram empregadas duas webcams como uma alternativa acessível e barata em relação à câmeras de alta precisão para medir a pose de uma Plataforma de Stewart utilizando três marcadores fiduciais. Foram fixados três marcadores ArUco em uma Plataforma de Stewart e foi realizada a medição da pose com o sistema estéreo desenvolvido. Inicialmente as câmeras foram calibradas utilizando um \textit{checkerboard} e algoritmos da biblioteca OpenCV para obter os parâmetros de câmera a serem utilizados no processamento das imagens. A detecção e cálculo de pose com os marcadores fiduciais é realizada pela biblioteca ArUco, usando uma aplicação em ROS em C++ adaptada para utilizar duas webcams. Foram aplicadas duas estratégias para a medição de pose, uma utilizando a biblioteca ArUco e outra adicionando um algoritmo de fotogrametria. Um sistema de medição externo com uma estação total foi usado para realizar a correlação entre os sistemas de coordenadas da Plataforma de Stewart e do sistema estéreo desenvolvido. Ambas as estratégias se mostraram equiparáveis e resultaram em erros médios de translação inferiores a 13 mm, e de orientação inferiores a 1.1 graus, em relação à posição nominal da Plataforma de Stewart.

 \textbf{Palavras chave} : ROS, ROS control, comunicaçao serial.


 %\tableofcontents


\chapter{Introdução}

O objetivo desta pesquisa é o projeto de um sistema de controle e comunicação baseado em ROS para um braço robô, fazendo ponte serial entre o computador que roda os programas em ROS e os drives que controlam os motores. (Inserir diagrama dos programas)

O projeto é baseado em pesquisas feitas nos , incluindo os trabalhos de (referenciar Priscila, Paulo, etc).

\chapter{Hardware}

\section{Computador}
\section{SiNet Hub 8}
\section{Drive ST5 Plus}
\section{Motores}

\chapter{Software}

\section{ROS}
\subsection{ROS control}
\subsection{moveit}
\subsection{serial bridge}
\subsection{pyconvo}
\section{ROS}



%Exemplo de citação \cite{gallimore2018robot} \cite{gallimore2018robot}

%\bibliographystyle{plain}
\bibliography{references}

\end{document}
